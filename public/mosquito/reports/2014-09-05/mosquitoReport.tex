\documentclass{article}

\usepackage{graphicx}
\usepackage{multicol}
\usepackage[center]{titlesec}
\usepackage{geometry}
%\usepackage{mathtools}



%
%\setlength{\columnseprule}{0.4pt}
%\setlength{\footskip}{20pt}
\usepackage{fancyhdr}
\fancyhf{}
\fancyhead[C]{Joe Brew $\bullet$ FDOH-Alachua $\bullet$ Ben Brew}
\fancyfoot[C]{  $\bullet$ Mosquito Surveillance Report \bullet$  }
\renewcommand\headrulewidth{1pt}
\renewcommand\footrulewidth{1pt}
\pagestyle{fancy}

%

\setlength{\columnsep}{1.5cm}
%\setlength{\columnseprule}{0.4pt}

%\MakeOuterQuote{"}



\graphicspath{ {C:/Users/BrewJR/Documents/fdoh/public/mosquito/reports/2014-09-05} }

%the next two lines adjust the third, centered section of the exec sum
\def\changemargin#1#2{\list{}{\rightmargin#2\leftmargin#1}\item[]}
\let\endchangemargin=\endlist 

\usepackage{Sweave}
\begin{document}
\Sconcordance{concordance:mosquitoReport.tex:mosquitoReport.Rnw:%
1 36 1 1 0 17 1 1 20 82 1 1 88 1 3 12 1 1 27 1 2 19 1 1 24 1 2 12 1 1 %
30 1 2 14 1 1 98 1 2 10 1 1 13 1 2 11 1 1 22 1 4 1 1 1 22 1 3 39 1}


\title{\textbf{Mosquito Surveillance Report}}
\author{Joe Brew and Ben Brew}


\maketitle
\tableofcontents

\vspace{40mm}

\begin{center}
\includegraphics[width=2cm]{doh}
\end{center}





\newgeometry{margin=5cm}
\fancyhfoffset[E,O]{0pt}


\vspace*{30mm}
%------------------------------------------
\section*{Executive Summary}
\addcontentsline{toc}{section}{Executive Summary}
%------------------------------------------
\hrulefill




\begin{multicols}{2} 
\setkeys{Gin}{width=0.49\textwidth}


%------------------------------------------
\subsection*{Most Recent Collection}
%------------------------------------------

As predicted, mosquito numbers fell over recent weeks to 93 mosquitoes per trap at the most recent trap collection (August 27, 2014).  

\vfill
\columnbreak



%------------------------------------------
\subsection*{Forecast}
%------------------------------------------

We forecast that mosquito levels will remain in the low range over the next 11 days (approximately 85-105 mosquitoes per trap), with a 95\% confidence interval of 0 to 257 per trap.  



\end{multicols}
\setkeys{Gin}{width=1\textwidth}

\vspace{2mm}
%------------------------------------------
\subsection*{Predictive Model Validation}
%------------------------------------------


\begin{changemargin}{1.5cm}{1.5cm} 

At 93 specimens per trap, the most recent collection was lower than our prediction of 148, but remained well within the 90\% confidence range of 0-302.  

\end{changemargin}

\hrulefill


\newgeometry{margin=2.5cm}
\fancyhfoffset[E,O]{0pt}
%------------------------------------------
%\section*{Visual Overview}
%\addcontentsline{toc}{section}{Visual Overview}
%------------------------------------------


%------------------------------------------
\section*{Visual Overview}
\addcontentsline{toc}{section}{Visual Overview}
%------------------------------------------
\hrulefill

\begin{multicols}{2} 
\setkeys{Gin}{width=0.49\textwidth}



%------------------------------------------
\subsection*{Time}
\addcontentsline{toc}{subsection}{Time}
%------------------------------------------
The current mosquito population is lower than usual for this time of year.
\begin{center}
%\begin{figure}
\includegraphics{mosquitoReport-002}
%\caption{Yearly comparison}
%\end{figure}
\end{center}


\vfill
\columnbreak
%------------------------------------------
\subsection*{Space}
\addcontentsline{toc}{subsection}{Space}
%------------------------------------------
Mosquitoes were largely scattered throughout the county.

\includegraphics{mosquitoReport-003}

\end{multicols}
\setkeys{Gin}{width=1\textwidth}


%\hrulefill
\vspace{10mm}

\begin{multicols}{2} 
\setkeys{Gin}{width=0.49\textwidth}


\vfill
\columnbreak
%------------------------------------------
\subsection*{Normality}
\addcontentsline{toc}{subsection}{Normality}
%------------------------------------------
The most recent collection was at levels equivalent to approximately the 46th percentile of historical (2008-13) levels.

\includegraphics{mosquitoReport-004}

\vfill
\columnbreak



%------------------------------------------
\subsection*{Disease Vectors}
\addcontentsline{toc}{subsection}{Disease Vectors}
%------------------------------------------

No vector of any diseases has seen significant increases over the last few weeks.

\includegraphics{mosquitoReport-005}

\vfill
\newpage
\end{multicols}
\setkeys{Gin}{width=1\textwidth}

%------------------------------------------
\section*{Forecast}
\addcontentsline{toc}{section}{Forecast}
%------------------------------------------
\hrulefill
\vspace{5mm}

\noindent We use recursive, quadratic linear regression modelling to forecast the average number of mosquitoes per trap up to 15 days in advance.\footnote{We are actively experimenting with non-parametric approaches to improve modelling accuracy, and expect to have a modified KNN model with better results by late September.}  

\includegraphics{mosquitoReport-006}


\newpage

%------------------------------------------
\section*{Vectors of Disease by Location}
\addcontentsline{toc}{section}{Vectors of Disease by Location}
%------------------------------------------
\hrulefill
\vspace{5mm}

\includegraphics{mosquitoReport-007}


%------------------------------------------
\section*{Mosquito Types}
\addcontentsline{toc}{section}{Mosquito Types}
%------------------------------------------
\hrulefill
\vspace{5mm}

No species some abnormal increases in recent weeks.  

\includegraphics{mosquitoReport-008}


\includegraphics{mosquitoReport-009}




% \begin{multicols}{2} 
% \setkeys{Gin}{width=0.49\textwidth}

% \end{multicols}
% \setkeys{Gin}{width=1\textwidth}
% \end{adjustwidth*}





\newpage
%------------------------------------------
\section*{Details of Predictive Model}
\addcontentsline{toc}{section}{Details of Predictive Model}
%------------------------------------------
\hrulefill
\vspace{5mm}

Historically, the model has performed well, correctly predicting the late summer spikes in 2012 and 2013.  Given the preference for accuracy at high numbers, the model intentionally includes outlying high observations, thereby weighting them. \\

Having simulated more than 65,000 unique models, our best fit equation (using the sum of least squares approach) was: 

\begin{center} 
$\hat{Y} = \beta_0+ \beta_1^2 (5.6508)$ + \beta_2 (0.5938)$ 
\end{center}

\noindent where $\hat{Y}$ is the estimated mean number of mosquitoes per trap, $\beta_0$ is set to 0, $\beta_1$ is the cumulative rainfall in the period 15 to 29 days prior to the date of prediction and $\beta_2$ is the mean number of mosquitoes per trap in the most recent prior trap collection.  \\

Though an original model relied only on rainfall, incorporating the most recent trap predicition saw our R-squared improve from 0.52 to 0.82.  This means that we can now explain over 80\% of the variance in mosquito populations up to 15 days ahead of time.  





\end{document}
